\documentclass{article}
\usepackage{amsmath}
\usepackage{amssymb}

\title{Solution to the 3n+m optimisation problem}
\author{Author: marzeq}
\date{Date: 2024-12-13}

\begin{document}

\maketitle

\section*{Problem}

\textbf{Let:}
\begin{itemize}
  \item \( \mathbf{a} = (a_x, a_y) \) and \( \mathbf{b} = (b_x, b_y) \) be vectors with known values \( \in \mathbb{N_+} \).
  \item \( \mathbf{p} = (p_x, p_y) \) be a point with known values \( \in \mathbb{N_+} \);
  \item \( n,m \in \mathbb{N_+} \).
\end{itemize}
The system of equations is as follows:
\[
na_x + mb_x = p_x
\]
\[
na_y + mb_y = p_y
\]
We are tasked with finding the lowest value of \( 3n+m \in \mathbb{N_+} \).

\section*{Solution}

\textbf{Let:}
\[
A = \begin{pmatrix}
a_x & a_y \\
b_x & b_y
\end{pmatrix}
\]
Rewrite the system as a matrix equation:
\[
A \begin{pmatrix}
n \\
m
\end{pmatrix}
= \begin{pmatrix}
p_x \\
p_y
\end{pmatrix}
\]

\subsection*{Case 1: If \( \det(A) \neq 0 \)}

Because \( \det(A) \neq 0 \), we can rewrite to move \(\begin{pmatrix}n\\ m\end{pmatrix}\) to the LHS:
\[
\begin{pmatrix}
n \\
m
\end{pmatrix}
= A^{-1} \begin{pmatrix}
p_x \\
p_y
\end{pmatrix}
\]
Then:
\[
n = \frac{b_yp_x - a_yp_y}{\det(A)}
\]
\[
m = \frac{-b_xp_x + a_xp_y}{\det(A)}
\]
These values are guaranteed to give the lowest possible value of \( 3n+m \), as they are the only pair that satisfies the system.
If \( n,m \notin \mathbb{N_+} \), there is no solution because these do not satisfy the initial requirement, and we cannot pick another pair.

\subsection*{Case 2: If \( \det(A) = 0 \)}

\subsubsection*{2.1 No pairs \( n, m \) satisfy the system}

For this to happen, \( \mathbf{p} \) must not satisfy \( \text{rank}([A | \mathbf{p}]) = \text{rank}(A) \).

\subsubsection*{2.2 Infinitely many pairs \( n, m \) satisfy the system}

Assuming that \( \mathbf{p} \) satisfies \( \text{rank}([A | \mathbf{p}]) = \text{rank}(A) \), we first check whether there exists at least one pair \( (n, m) \) where both \( n \) and \( m \) are in \( \mathbb{N}_+ \). This can be done by checking the particular solution \( \mathbf{v_p} = \begin{pmatrix} n_p \\ m_p \end{pmatrix} \).
\begin{itemize}
    \item If both \( n_p \) and \( m_p \) are positive integers, we proceed to find the general solution.
    \item If either \( n_p \) or \( m_p \) is not a positive integer, the system has no solution \( n,m \in \mathbb{N}_+ \).
\end{itemize}
If a valid pair \( (n_p, m_p) \) is found, the general solution is given by:
\[
\mathbf{v} = \mathbf{v_p} + t\mathbf{v_n}
\]
where \( \mathbf{v_p} \) is a particular solution, \( \mathbf{v_n} \) is a vector in the null space of \( A \), and \( t \in \mathbb{Z} \).
The objective function to minimize is:
\[
f(t) = 3n_p + m_p + t(3n_n + m_n)
\]
where \( (n_p, m_p) \) is a particular solution and \( (n_n, m_n) \) is a basis vector for the null space.
To minimize \( f(t) \), perform the following steps:

\begin{enumerate}
    \item Compute \( 3n_n + m_n \). If it equals 0, then \( f(t) \) is constant, and all solutions have the same value. In this case, yield all pairs \( n,m \).
    
    \item If \( 3n_n + m_n \neq 0 \), find the integer \( t \) that minimizes \( f(t) \). This depends on the sign of \( 3n_n + m_n \):
    \begin{itemize}
        \item If \( 3n_n + m_n > 0 \), decrease \( t \) until the result no longer satisfies the integer constraint (i.e., \( n_p + t n_n > 0 \) and \( m_p + t m_n > 0 \)). Yield that pair.
        \item If \( 3n_n + m_n < 0 \), increase \( t \) similarly and yield the resulting pair.
    \end{itemize}
\end{enumerate}

\end{document}

