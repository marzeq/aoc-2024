\documentclass{article}
\usepackage{amsmath}
\usepackage{amssymb}
\setlength{\parindent}{0pt}
\newcommand{\intpos}{\mathbb{N}_+}

\title{Solution to the \(3n + m\) Optimisation Problem}
\author{Author: marzeq}
\date{Date: 2024-12-13}

\begin{document}

\maketitle

\section*{Problem}

\textbf{Let:}
\begin{itemize}
  \item \( \mathbf{a} = (a_x, a_y) \) and \( \mathbf{b} = (b_x, b_y) \) be vectors with known values \( \in \intpos \).
  \item \( \mathbf{P} = (P_x, P_y) \) be a point with known values \( \in \intpos \);
  \item \( n,m \in \intpos \).
\end{itemize}

The system of equations is as follows:
\[
  \begin{cases}
  a_xn + b_xm = P_x \\
  a_yn + b_ym = P_y
  \end{cases}
\]

We are tasked with finding the smallest value of \( 3n + m \) such that \( n, m \in \intpos \).

\section*{Solution}

\textbf{Let:}
  \[
  A = \begin{pmatrix}
  a_x & a_y \\
  b_x & b_y
  \end{pmatrix}
\]

Rewrite the system as a matrix equation:
\[
  A \begin{pmatrix}
  n \\
  m
  \end{pmatrix}
  = \begin{pmatrix}
  P_x \\
  P_y
  \end{pmatrix}
\]

\subsection*{Case 1: If \( \det(A) \neq 0 \)}

Because \( \det(A) \neq 0 \), we can apply Cramér's rule:
\[
  n = \frac{\det\begin{pmatrix} P_x & a_y \\ P_y & b_y \end{pmatrix}}{\det(A)} = \frac{b_yP_x - a_yP_y}{\det(A)}
\]
\[
  m = \frac{\det\begin{pmatrix} a_x & P_x \\ b_x & P_y \end{pmatrix}}{\det(A)} = \frac{a_xP_y - b_xP_x}{\det(A)}
\]

Therefore, as long as \( n, m \in \intpos \) (to satisfy the initial requirement) the value we are looking for is:
\[
  \frac{P_x(3b_y - b_x) + P_y(-3a_y + a_x)}{\det(A)}
\]

\subsection*{Case 2: If \( \det(A) = 0 \)}

We can rewrite \( m \) in terms of \( n \) (using the first equation, for example):
\[
  m = \frac{P_x - n a_x}{b_x}
\]

Now, substitute into \( 3n + m \):
\[
  3n + m = 3n + \frac{P_x - n a_x}{b_x} = \left( \frac{3 b_x - a_x}{b_x} \right) n + \frac{P_x}{b_x}
\]

The result is a linear function of the form \( f(n) = an + b \), where \( a, b > 0 \). We now want to minimize \( f(n) \) subject to the constraint that it must be a positive integer.\
For that to be true, the numerator \( P_x - n a_x \) must be divisible by \( b_x \). This gives the following condition:
\[
P_x - n a_x \equiv 0 \pmod{b_x}
\]
or equivalently:
\[
n a_x \equiv P_x \pmod{b_x}
\]
So, we need to find the smallest \( n \in \intpos \) that satisfies the linear congruence:
\[
n a_x \equiv P_x \pmod{b_x}
\]

Thus, the value we are looking for is \( f(n) \) of the previously mentioned \( n \).
\end{document}

